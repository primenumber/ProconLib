\section{Structure}

\subsection{Union-Find Tree}
互いに素な集合を扱うためのデータ構造であり,計算量は $O(\alpha(N))$ である.
経路圧縮のみを行い実装を軽くした場合,計算量は $O(\log N)$ である.
\lstinputlisting{../Structure/UnionFind.cpp}

\subsection{Segment Tree}
デフォルトは Range Minimum Query.区間は半開区間で指定すること.
\lstinputlisting{../Structure/SegmentTree.cpp}
デフォルトは区間に足せる Range Sum Query と Range Minimum Query.遅延伝播セグメント木である.
\lstinputlisting{../Structure/SegmentTreeLazy.cpp}
遅延伝播セグメント木の強いバージョン
\lstinputlisting{../Structure/SegmentTreeLazyUniversal.cpp}
必要な頂点だけ作って処理する RMQ.デフォルトでは $i$ 番目の要素は $i+1$ で初期化されている.
\lstinputlisting{../Structure/BigRangeQuery.cpp}
二次元セグメント木
\lstinputlisting{../Structure/SegmentTree2D.cpp}

\subsection{Fenwick Tree}
FenwickTree.デフォルトは Range Sum Query であり,
引数として与えられるindex以下の部分の和を求める.
\lstinputlisting{../Structure/FenwickTree.cpp}

\subsection{Square Root Decomposition}
平方分割によるRMQ.区間の回転や値の挿入や削除ができる.
区間は閉区間で指定すること.
\lstinputlisting{../Structure/SqrtDecomposition.cpp}

\subsection{Slide-Min}
スライド最小値.コンストラクタでスライド幅を指定すると,各クエリについて直前$w$個の最小値をO(1)で返す.
\lstinputlisting{../Structure/SlideMin.cpp}

\subsection{Convex Hull Trick}
\lstinputlisting{../Structure/ConvexHullTrick.cpp}

\subsection{Skew Heap}
meldが $O(\log N)$ でできる priority queue.
\lstinputlisting{../Structure/SkewHeap.cpp}

\subsection{Treap}
Treap.乱択アルゴリズムを用いた平衡二分木である.
\lstinputlisting{../Structure/Treap.cpp}

\subsection{Partition Tree}
$O(N \log^2 N)$ の前処理で,区間のk-th elementを $O(\log^2 N)$ で求めることができる.
\lstinputlisting{../Structure/PartitionTree.cpp}

\subsection{kD-Tree}
$O(N \log N)$ の前処理で,各クエリに対して $O(\log N)$ で与えられた点に最も近い点を求めることができる.
与えられた領域内の点の列挙を行うこともできる.
\lstinputlisting{../Structure/kdTree.cpp}
